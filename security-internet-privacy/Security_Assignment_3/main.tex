\documentclass{article}
\usepackage{graphicx} % Required for inserting images
\usepackage{float}

\title{Security Assignment 3: Internet Privacy}
\author{Chris Low}

\begin{document}

\maketitle

\section{Methodology (unencrypted DNS)}

I ran this experiment on the UChicago WiFi network using a Mac laptop.  
Before visiting Reddit, I closed other browser tabs and cleared the cache in Chrome so that the page load would look as fresh as possible.

I then started Wireshark on the active interface and set a display filter of

\begin{verbatim}
dns
\end{verbatim}

so only DNS packets would be shown. With the capture running, I typed
\texttt{https://www.reddit.com} in the address bar and waited for the
front page to finish loading. After the bursts of DNS traffic died down,
I stopped the capture and saved it as a pcap file that contained only DNS
packets.

In this trace the client IP is \texttt{10.150.63.208}, and the main DNS
resolver is \texttt{128.135.249.50}, which belongs to UChicago.  
There are several dozen DNS queries and responses in total. Screenshots of the packet capture is shown below, but you can also find them in the file \verb|reddit_from_dns.pcap|.

\begin{figure}[H]
    \centering
    \includegraphics[width=1\linewidth]{DNS_capture_P1.png}
    \label{fig:placeholder}
\end{figure}

\begin{figure}[H]
    \centering
    \includegraphics[width=1\linewidth]{DNS_capture_P2.png}
    \label{fig:placeholder}
\end{figure}

\section{Who can see that I visited Reddit from unencrypted DNS}

Because these DNS packets are unencrypted, several parties can see the
domain names in clear text.

\subsection*{Campus DNS resolver and local network}

All DNS queries in the trace go from \texttt{10.150.63.208} (my machine)
to \texttt{128.135.249.50}. This means the UChicago recursive DNS
resolver sees every domain that I look up. The resolver can easily tell
that I visited Reddit from queries such as:

\begin{itemize}
  \item \texttt{www.reddit.com}
  \item \texttt{www.redditstatic.com}
  \item \texttt{i.redd.it}
  \item \texttt{preview.redd.it}
  \item \texttt{styles.redditmedia.com}
  \item \texttt{matrix.redditspace.com}
  \item \texttt{w3-reporting-nel.reddit.com} and \texttt{w3-reporting-csp.reddit.com}
\end{itemize}

Any administrator who operates this resolver, or who can monitor the
UChicago internal network, can link those names to my client IP and to
the time of the visit.

If the campus resolver forwards queries to an upstream ISP or another
provider, that upstream resolver would also see the same domain names
and could draw the same conclusion.

\subsection*{Authoritative DNS and hosting providers}

Authoritative DNS servers for Reddit and for its infrastructure
providers also see some information. For example, the trace includes
queries for Akamai domains such as

\begin{itemize}
  \item \texttt{a1961.g2.akamai.net}
  \item \texttt{e6858.dsce9.akamaiedge.net}
\end{itemize}

These belong to Akamai, which provides Reddit with static assets and media upon some Google search. The authoritative name servers for these domains receive the DNS
queries and can see that a UChicago address is trying to reach hosts
that serve Reddit content.

Unlike the campus resolver, they do not see the full set of sites I
visit, but they do see that I am contacting infrastructure that belongs
to Reddit.

\section{Other entities that know I visited Reddit}

Even if we ignore the DNS layer, other parties learn about my visit as
part of normal web operation. I grouped the domains in the trace by
company, based on the hostnames visible in the DNS packets.

\subsection*{Reddit}

Reddit related domains in the capture include:

\begin{itemize}
  \item \texttt{www.reddit.com} (main site)
  \item \texttt{i.redd.it}, \texttt{preview.redd.it} (images and link previews)
  \item \texttt{www.redditstatic.com}, \texttt{styles.redditmedia.com} (static assets and styles)
  \item \texttt{gql-realtime.reddit.com}, \texttt{matrix.redditspace.com}
  \item \texttt{w3-reporting-nel.reddit.com}, \texttt{w3-reporting-csp.reddit.com} (reporting endpoints)
\end{itemize}

Once the DNS lookup succeeds, my browser opens HTTPS connections to
these hosts. Reddit therefore knows that someone at a UChicago address
visited the front page, which APIs were used, and which images and
scripts were loaded. If I had been logged in, Reddit could tie this to
my account.

\subsection*{Google}

Several Google owned domains appear in the trace:

\begin{itemize}
  \item \texttt{safebrowsing.googleapis.com}
  \item \texttt{www.google.com} and \texttt{accounts.google.com}
  \item \texttt{encrypted-tbn0.gstatic.com}
  \item \texttt{optimizationguide-pa.googleapis.com}
  \item \texttt{clients4.google.com} and other \texttt{googleapis.com} hostnames
\end{itemize}

Chrome uses Google Safe Browsing to check URLs against a phishing and
malware list, so contacting \texttt{safebrowsing.googleapis.com} tells
Google that the browser is checking a new site. Some of the other
\texttt{googleapis.com} and \texttt{gstatic.com} hostnames serve fonts,
static files or browser services. If Reddit embeds any Google services
or if Chrome talks to Google while the page loads, Google can infer that
the browser just loaded Reddit, even though the page itself is hosted
elsewhere.

\subsection*{Apple}

The trace includes several Apple hostnames, for example:

\begin{itemize}
  \item \texttt{captive.g.aaplimg.com}
  \item \texttt{configuration.ls.apple.com}
  \item \texttt{swallow-apple.com}, \texttt{fbs.smoot.apple.com},
        \texttt{smoot-feedback.v.aaplimg.com}
  \item \texttt{safebrowsing-proxy.g.aaplimg.com}
\end{itemize}

These are not caused directly by Reddit. They come from macOS itself.
For instance, \texttt{captive.g.aaplimg.com} is used to detect captive
portals and to check internet connectivity. Apple therefore learns that
this Mac is online and periodically making these checks. When combined
with timing and other signals, Apple could correlate this with general
browsing activity.

\subsection*{Akamai and other CDNs}

As mentioned above, Akamai appears through domains such as
\texttt{a1961.g2.akamai.net} and \texttt{e6858.dsce9.akamaiedge.net}.
These hosts serve static assets for Reddit. When my browser loads
images, style sheets or JavaScript from Akamai, Akamai sees my IP
address, the Reddit related hostnames, and the time of access. Even if
they do not see DNS for \texttt{www.reddit.com} itself, they can still
see that they are serving Reddit content to a client at UChicago.

\subsection*{Slack and other background services}

There are DNS queries for \texttt{slack.com} as well. These are almost
certainly from the Slack desktop app running in the background, not from
Reddit itself. They still show up in the capture because Wireshark is
recording all DNS traffic on the interface. This illustrates that a DNS
trace taken on a real machine often contains noise from other
applications.

\section{Privacy concerns by company}

The privacy risks are different for each of these parties.

\subsection*{UChicago DNS and any upstream ISP}

The campus resolver at \texttt{128.135.249.50} sees every domain that my
machine looks up, in clear text, along with timestamps and my client IP.
Over time this data can reveal a very detailed picture of my browsing
habits and daily routine. Policies at a university are usually more
benign than at a commercial ISP, but technically the capability is the
same. Logs could be kept for troubleshooting, but they could also be
used for monitoring or handed over if requested.

\subsection*{Reddit}

Reddit needs to know which pages I visit, otherwise it cannot serve the
site. Still, Reddit can log all of my page views, the subreddits I look
at, my interactions, and my approximate location from IP. If I am
logged in, this history can be tied to my account and used for
recommendations, content ranking or targeted advertising. If the logs
are ever breached or shared, this could reveal a lot about my interests.

\subsection*{Google}

Because Chrome and some page components talk to Google services during
the visit, Google can often infer that I am on Reddit, even though I am
not on a Google website. Google already has search history, YouTube
watch history and other signals. Adding another source of browsing data
lets Google build a more complete profile that can be used for ads or
other personalization. The concern here is cross site tracking and the
concentration of data in one company.

\subsection*{Akamai and other CDNs}

Akamai mainly sees requests for static resources. On its own, this is
less sensitive than full page content, but Akamai serves a very large
fraction of the web. If it wanted to, it could correlate traffic from
the same IP across different customer sites and learn which large
services someone uses. Even if the company does not do this in practice,
the potential is there because unencrypted DNS and hostnames in HTTPS
requests leak quite a bit of information.

\subsection*{Apple}

Apple learns that the device is online and that it is contacting Apple
network check and feedback services. This is not specific to Reddit, but
it adds to the general telemetry Apple has about the device. The concern
is less about this one site, and more about the accumulation of many
small signals from different services on the system.

\subsection*{Other background services}

Queries from Slack and similar apps are a reminder that DNS captures
reflect the whole device, not just one browser tab. Any service that
runs in the background can add entries to the trace and can also be
logged by the DNS resolver. This widens the amount of information that a
network operator or DNS provider can collect about a user.


\vspace{5mm}
Overall, the unencrypted DNS trace for a single visit to
\texttt{reddit.com} exposes not only that I visited Reddit, but also
which third party services and infrastructure providers were involved,
and which other applications on my machine were active at the same time.




\section*{Part 2: Encrypted DNS Results}

After switching Chrome to use Cloudflare (1.1.1.1) for secure DNS, the DNS
traffic in Wireshark changed sharply. In the unencrypted trace, there
were many DNS packets for Reddit domains and for several third party
services. In contrast, the encrypted DNS trace contained almost no DNS
queries from the browser. The only DNS packets that appeared were from
macOS background services, such as Apple network checks, and a small
number of CDN related lookups. No Reddit related domain names appeared
in the capture, as shown: 

\begin{figure} [H]
    \centering
    \includegraphics[width=1\linewidth]{encrypted_dns.png}
    \label{fig:placeholder}
\end{figure}

\subsection*{Who can still see that I visited Reddit}

Encrypted DNS removes visibility from the campus network and anyone else
watching local DNS traffic. UChicago’s resolver no longer receives my
DNS queries, and the domain names are no longer exposed on the wire. 

\textbf{The new party that gains visibility is Cloudflare}, since Chrome now sends
its DNS lookups directly to Cloudflare over an encrypted connection.
Cloudflare can see every domain the browser resolves, including all of
the Reddit related names that were visible to UChicago in the first
trace.

Other types of entities still see my activity even with encrypted DNS.
\textbf{Reddit} sees the visit through my HTTPS connections to its servers. Its
\textbf{CDNs, such as Akamai or Fastly,} still see requests for images, scripts,
and other static files. These connections reveal that I am loading Reddit
content even though the DNS lookups do not appear in Wireshark.

\subsection*{Privacy tradeoffs}

Encrypted DNS shifts who can observe my browsing. It prevents the campus
network, local administrators, and passive observers on the WiFi from
reading my DNS queries. This removes one of the easiest ways to monitor
which sites I visit.

However, it does not hide my activity from every party. \textbf{The DNS
resolver I choose, in this case Cloudflare, now receives all of my
browser’s DNS traffic.} Cloudflare can build the same type of browsing
profile that UChicago could build in the unencrypted case, although
Cloudflare may have different policies about logging and retention.

\textbf{Encrypted DNS also does not hide information from the sites I connect
to. Reddit still sees my HTTPS requests.} Its CDNs still see that my
browser is fetching Reddit assets. These services can infer that I am
visiting Reddit from the hostnames inside the encrypted HTTPS requests.

\textbf{In summary, encrypted DNS protects my traffic from local observers but
moves trust to the DNS provider and does not prevent the destination
sites or their CDNs from seeing that I visited them.}



\end{document}
